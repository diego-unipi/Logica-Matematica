\section{Sintassi}

\begin{definition}[Formula in $\mathsf{ND}$]
    Sia $L$ un linguaggio, una \vocab{$L$-formula in $\mathsf{ND}$} è una sequenza finita di simboli nell'alfabeto $\{\land, \lor, \to, \neg, \forall, \exists, \top, \bot\} \cup L$ (il primo insieme è costituito dai \vocab{simboli logici}).
\end{definition}

\begin{definition}[Dimostrazione in $\mathsf{ND}$]
    Una \vocab{dimostrazione in $\mathsf{ND}$} di una $L$-formula $\varphi$ è una sequenza di $L$-formule $\varphi_1, \varphi_2, \ldots, \varphi_n$ tale che $\varphi_n = \varphi$ e per ogni $1 \leq k \leq n$,
    o $\varphi_k$ è un assioma, oppure $\varphi_k$ è in $T$, oppure $\varphi_k$ è ottenuta da formule precedenti tramite una \vocab{regola di inferenza} del \vocab{sistema di deduzione naturale} (\vocab{$\mathsf{ND}$}) qui di seguito riportate.
\end{definition}

\subsection{\texorpdfstring{Sistema di deduzione naturale ($\mathsf{ND}$)}{Sistema di deduzione naturale (ND)}}
\iffalse
$$
\begin{array}{l}
%Ax, Wk
\AxiomC{$\vphantom{T}$}
\LeftLabel{\textcolor{purple}{Ax}}
\UnaryInfC{$\varphi \vdash \varphi$}
\DisplayProof
\qquad
\AxiomC{$T \vdash \varphi$}
\LeftLabel{\textcolor{purple}{Wk}}
\RightLabel{\quad con $T \subseteq T'$}
\UnaryInfC{$T' \vdash \varphi$}
\DisplayProof \\[4ex]
%TOP, BOT
\AxiomC{$\vphantom{T}$}
\LeftLabel{$\textcolor{green}{\mathrm{In}}_{\top}$}
\UnaryInfC{$T \vdash \top$}
\DisplayProof
\qquad
\AxiomC{$T \vdash \textcolor{purple}{\bot}$}
\LeftLabel{$\textcolor{orange}{\mathrm{El}}_{\bot}$}
\UnaryInfC{$T \vdash \varphi$}
\DisplayProof \\[4ex]

%AND
\AxiomC{$T \vdash \varphi$}
\AxiomC{$T \vdash \psi$}
\LeftLabel{$\textcolor{green}{\text{In}}_{\land}$}
\BinaryInfC{$T \vdash \varphi \,\textcolor{purple}{\land} \, \psi$}
\DisplayProof
\qquad
\AxiomC{$T \vdash \varphi \,\textcolor{purple}{\land} \, \psi$}
\LeftLabel{$\textcolor{orange}{\text{El}}_{\land\cdot}$}
\UnaryInfC{$T \vdash \varphi$}
\DisplayProof
\qquad
\AxiomC{$T \vdash \varphi \,\textcolor{purple}{\land} \, \psi$}
\LeftLabel{$\textcolor{orange}{\text{El}}_{\cdot\land}$}
\UnaryInfC{$T \vdash \psi$}
\DisplayProof \\[4ex]

%OR
\AxiomC{$T \vdash \varphi$}
\LeftLabel{$\textcolor{green}{\mathrm{In}}_{\lor\cdot}$}
\UnaryInfC{$T \vdash \varphi \,\textcolor{purple}{\lor} \, \psi$}
\DisplayProof
\qquad
\AxiomC{$T \vdash \psi$}
\LeftLabel{$\textcolor{green}{\mathrm{In}}_{\cdot\lor}$}
\UnaryInfC{$T \vdash \varphi \,\textcolor{purple}{\lor} \, \psi$}
\DisplayProof
\qquad
\AxiomC{$T \vdash \varphi \,\textcolor{purple}{\lor} \, \psi$}
\AxiomC{$T, \varphi \vdash \theta$}
\AxiomC{$T, \psi \vdash \theta$}
\LeftLabel{$\textcolor{orange}{\mathrm{El}}_{\lor}$}
\TrinaryInfC{$T \vdash \theta$}
\DisplayProof \\[4ex]

%IMPLICATION
\AxiomC{$T, \varphi \vdash \psi$}
\LeftLabel{$\textcolor{green}{\mathrm{In}}_{\to}$}
\UnaryInfC{$T \vdash \varphi \,\textcolor{purple}{\to} \, \psi$}
\DisplayProof
\qquad
\AxiomC{$T \vdash \varphi \,\textcolor{purple}{\to} \, \psi$}
\AxiomC{$T \vdash \varphi$}
\LeftLabel{$\textcolor{orange}{\mathrm{El}}_{\to}$}
\RightLabel{(Modus Ponens)}
\BinaryInfC{$T \vdash \psi$}
\DisplayProof
\\[4ex]

%NOT, RaA
\AxiomC{$T, \varphi \vdash \,\textcolor{purple}{\bot}$}
\LeftLabel{$\textcolor{green}{\mathrm{In}}_{\neg}$}
\UnaryInfC{$T \vdash\textcolor{purple}{\neg}\, \varphi$}
\DisplayProof
\qquad
\AxiomC{$T \vdash \textcolor{purple}{\neg}\, \varphi$}
\AxiomC{$T \vdash \varphi$}
\LeftLabel{$\textcolor{orange}{\mathrm{El}}_{\neg}$}
\BinaryInfC{$T \vdash \psi$}
\DisplayProof
\qquad
\AxiomC{$T, \textcolor{purple}{\neg}\, \varphi \vdash \,\textcolor{purple}{\bot}$}
\LeftLabel{$\textcolor{purple}{\text{RaA}}$}
\UnaryInfC{$T \vdash \varphi$}
\DisplayProof
\\[4ex]

%FORALL
\AxiomC{$T \vdash \varphi$}
\LeftLabel{$\textcolor{green}{\mathrm{In}}_{\forall}$}
\RightLabel{(con $\textcolor{purple}{x_k} \not \in \vl(T)$)}
\UnaryInfC{$T \vdash \, \textcolor{purple}{\forall x_k}\, \varphi$}
\DisplayProof
\qquad
\AxiomC{$T \vdash \, \textcolor{purple}{\forall x_k}\, \varphi$}
\LeftLabel{$\textcolor{orange}{\mathrm{El}}_{\forall}$}
\UnaryInfC{$T \vdash\varphi[t/\textcolor{purple}{x_k}]$}
\DisplayProof
\\[4ex]

%EXISTS
\AxiomC{$T \vdash \varphi[t/\textcolor{purple}{x_k}]$}
\LeftLabel{$\textcolor{green}{\mathrm{In}}_{\exists}$}
\UnaryInfC{$T \vdash \, \textcolor{purple}{\exists x_k}\, \varphi$}
\DisplayProof
\qquad
\AxiomC{$T \vdash \, \textcolor{purple}{\exists x_k}\, \varphi$}
\AxiomC{$T, \varphi \vdash \psi$}
\RightLabel{(con $\textcolor{purple}{x_k} \not \in \vl(T, \psi)$)}
\LeftLabel{$\textcolor{orange}{\mathrm{El}}_{\exists}$}
\BinaryInfC{$T \vdash \psi$}
\DisplayProof \\[4ex]

%EQUALITY
\AxiomC{$\vphantom{T}$}
\LeftLabel{$\textcolor{green}{\mathrm{In}}_{=}$}
\UnaryInfC{$T \vdash t \, \textcolor{purple}{=} \, t$}
\DisplayProof
\qquad
\AxiomC{$T \vdash s \, \textcolor{purple}{=} \, t$}
\AxiomC{$T \vdash \varphi[s/\textcolor{purple}{x_k}]$}
\LeftLabel{$\textcolor{orange}{\mathrm{El}}_{=}$}
\BinaryInfC{$T \vdash \varphi[t/\textcolor{purple}{x_k}]$}
\DisplayProof

\end{array}
$$
\fi
%%%%%%%%%%%%%%%%%%%%%%%%%%%%%%%%%%%%%%%%%%%%%%%%%%%%%%%%%%%%%%%%%%%%%%%%%%%%%%%

\begin{definition}[Sequente in $\mathsf{ND}$]
    Una \vocab{sequente} $T \vdash_{\mathsf{ND}} \varphi$ indica che esiste una dimostrazione in $\mathsf{ND}$ di $\varphi$ a partire dalle formule in $T$.
\end{definition}

\begin{definition}[Formula in $\mathsf{ND}_{\to, \bot, \exists, =}$]
    Sia $L$ un linguaggio, una \vocab{$L$-formula in $\mathsf{ND}_{\to, \bot, \exists, =}$} è una sequenza finita di simboli nell'alfabeto $\{\to, \bot, \exists, =\} \cup L$ (il primo insieme è costituito dai \vocab{simboli logici} di $\mathsf{ND}_{\to, \bot, \exists, =}$).
\end{definition}

\begin{definition}[Dimostrazione in $\mathsf{ND}_{\to, \bot, \exists, =}$]
    Una \vocab{dimostrazione in $\mathsf{ND}_{\to, \bot, \exists, =}$} di una $L$-formula $\varphi$ è una sequenza di $L$-formule $\varphi_1, \varphi_2, \ldots, \varphi_n$ tale che $\varphi_n = \varphi$ e per ogni $1 \leq k \leq n$,
    o $\varphi_k$ è un assioma, oppure $\varphi_k$ è in $T$, oppure $\varphi_k$ è ottenuta da formule precedenti tramite una \vocab{regola di inferenza} del \vocab{sistema di deduzione naturale ridotto} (\vocab{$\mathsf{ND}_{\to, \bot, \exists, =}$}) qui di seguito riportate.
\end{definition}

\subsection{\texorpdfstring{Sistema ridotto ($\mathsf{ND}_{\to, \bot, \exists, =}$)}{Sistema ridotto (ND→,⊥,∃,=)}}
\iffalse
$$
\begin{array}{l}
%Ax, Wk, RaA
\AxiomC{$\vphantom{T}$}
\LeftLabel{\textcolor{purple}{Ax}}
\UnaryInfC{$T \vdash \varphi$}
\DisplayProof
\qquad
\AxiomC{$T \vdash \varphi$}
\LeftLabel{\textcolor{purple}{Wk}}
\UnaryInfC{$T \vdash \varphi$}
\DisplayProof
\qquad
\AxiomC{$T, \textcolor{purple}{\neg}\, \varphi \vdash \,\textcolor{purple}{\bot}$}
\LeftLabel{\textcolor{purple}{\text{RaA}}}
\UnaryInfC{$T \vdash \varphi$}
\DisplayProof   \\[4ex]

%IMPLICATION
\AxiomC{$T, \varphi \vdash \psi$}
\AxiomC{$T \vdash \varphi \to \psi$}
\LeftLabel{$\textcolor{green}{\text{In}}_{\to}$}
\BinaryInfC{$T \vdash \varphi \to \psi$}
\DisplayProof
\qquad
\AxiomC{$T \vdash \varphi \to \psi$}
\AxiomC{$T \vdash \varphi$}
\LeftLabel{$\textcolor{orange}{\mathrm{El}}_{\to}$}
\RightLabel{(Modus Ponens)}
\BinaryInfC{$T \vdash \psi$}
\DisplayProof \\[4ex]

%EXISTS
\AxiomC{$T \vdash \varphi[t/\textcolor{purple}{x_k}]$}
\LeftLabel{$\textcolor{green}{\text{In}}_{\exists}$}
\UnaryInfC{$T \vdash \exists x_k\,\varphi$}
\DisplayProof
\qquad
\AxiomC{$T \vdash \exists x_k\,\varphi$}
\LeftLabel{$\textcolor{orange}{\mathrm{El}}_{\exists}$}
\UnaryInfC{$T \vdash \varphi[t/x_k]$}
\DisplayProof \\[4ex]

%EQUALITY
\AxiomC{$\vphantom{T}$}
\LeftLabel{$\textcolor{green}{\mathrm{In}}_{=}$}
\UnaryInfC{$T \vdash t \,\textcolor{purple}{=}\, t$}
\DisplayProof
\qquad
\AxiomC{$T \vdash s \,\textcolor{purple}{=}\, t$}
\AxiomC{$T \vdash \varphi[s/\textcolor{purple}{x_k}]$}
\LeftLabel{$\textcolor{orange}{\mathrm{El}}_{=}$}
\BinaryInfC{$T \vdash \varphi[t/x_k]$}
\DisplayProof

\end{array}
$$
\fi
%%%%%%%%%%%%%%%%%%%%%%%%%%%%%%%%%%%%%%%%%%%%%%%%%%%%%%%%%%%%%%%%%%%%%%%%%%%%%%%%

\begin{definition}[Sequente in $\mathsf{ND}_{\to, \bot, \exists, =}$]
    Un \vocab{sequente} $T \vdash_{\mathsf{ND}_{\to, \bot, \exists, =}} \varphi$ indica che esiste una dimostrazione in $\mathsf{ND}_{\to, \bot, \exists, =}$ di $\varphi$ a partire dalle formule in $T$.
\end{definition}

\begin{definition}[$\textcolor{purple}{\neg, \top, \land, \lor, \forall}$]
    Definiamo i seguenti simboli logici:
    \begin{alignat*}{4}
        \textcolor{purple}{\neg}\,\varphi &\overset{\text{def}}{=} \varphi \,\textcolor{purple}{\to}\, \textcolor{purple}{\bot} \qquad & \textcolor{purple}{\top} &\overset{\text{def}}{=} \textcolor{purple}{\neg\bot} \qquad\quad & \varphi \,\textcolor{purple}{\land}\, \psi &\overset{\text{def}}{=} \neg(\varphi \,\textcolor{purple}{\to}\, \neg\psi) \\
        \varphi \,\textcolor{purple}{\lor}\, \psi &\overset{\text{def}}{=} \neg\varphi \,\textcolor{purple}{\to}\, \psi \qquad & \textcolor{purple}{\forall x_k}\,\varphi &\overset{\text{def}}{=} \textcolor{purple}{\neg\,\exists x_k\,\neg}\,\varphi
    \end{alignat*}
\end{definition}

\begin{exercise}[$\mathsf{ND}_{\to, \bot, \exists, =}$ dimostra $\mathsf{ND}$]
    In $\mathsf{ND}_{\to, \bot, \exists, =}$ si possono dimostrare le regole mancanti di $\mathsf{ND}$ nel seguente ordine: 
    \textcolor{green}{In}$_{\neg}$, \textcolor{orange}{El}$_{\bot}$, \textcolor{orange}{El}$_{\neg}$, \textcolor{green}{In}$_{\top}$, \textcolor{green}{In}$_{\land}$, \textcolor{orange}{El}$_{\land\cdot}$, \textcolor{orange}{El}$_{\cdot\land}$, \textcolor{green}{In}$_{\lor\cdot}$, \textcolor{green}{In}$_{\cdot\lor}$,
    \textcolor{orange}{El}$_{\lor}$, \textcolor{green}{In}$_{\forall}$, \textcolor{orange}{El}$_{\forall}$. 
\end{exercise}

\begin{exercise}[Semantica di Tarski indotta]
    Verificare che la semantica di Tarski, già definita in precedenza, coincide con quella indotta dalle definizioni date sopra 
    per i simboli: $\textcolor{purple}{\neg, \top, \land, \lor, \forall}$.
\end{exercise}

\begin{notation}[Sequenti]
    Avendo verificato che $\mathsf{ND}$ e $\mathsf{ND}_{\to, \bot, \exists, =}$ sono equivalenti, d'ora in avanti indicheremo semplicemente con $T \vdash \varphi$ il sequente dimostrabile in entrambi i sistemi.
\end{notation}

\subsection{Teoremi di correttezza e completezza}

\begin{proposition}
    Data $T$ una $L$-teoria e $\varphi$ una $L$-formula, vale che $T \vdash \varphi \iff$ \textcolor{purple}{per ogni}
    $L$-struttura $M = (D;i)$ e \textcolor{purple}{per ogni} $v : \Var \to D$ valutazione delle variabili, si ha $M \models \{v\}T \implies M \models \{v\}\varphi$.
\end{proposition}

\begin{corollary}[Teorema di correttezza]
    Sia $T$ una $L$-teoria fatta da formule chiuse, e sia $\varphi$ una $L$-formula, allora:
    \[T \vdash \varphi \implies T \models \varphi\]
\end{corollary}

\begin{corollary}[Teorema di completezza]
    Sia $T$ una $L$-teoria fatta da formule chiuse, e sia $\varphi$ una $L$-formula, allora:
    \[T \models \varphi \implies T \vdash \varphi\]
\end{corollary}

