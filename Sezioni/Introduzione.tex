\section{Introduzione}
La \vocab{logica matematica} nasce dalle ricerche, a cavallo fra il XIX e il XX secolo, sui fondamenti della matematica. Sebbene diverse importanti intuizioni siano considerevolmente più antiche - per esempio il 
metodo assiomatico degli Elementi di Euclide, l'analisi delle leggi del pensiero nella Logica di Aristotele, l'idea del ragionamento simbolico di Leibniz - si può sostenere che la logica nasca, come branca della matematica, nel momento 
in cui i concetti quali la \vocab{dimostrazione} o la \vocab{relazione di conseguenza logica} vengono concepiti come \textcolor{orange}{oggetti matematici}. In altre parole, la logica diventa matematica nel momento in cui si rende conto che 
le proposizioni matematiche - scrivibili e scritte in un opportuno linguaggio simbolico - sono passibili di studio alla stessa maniera, per dire, dei polinomi o dei numeri interi. Per cui, per esempio,
distinguere quali numeri siano composti, quali polinomi abbiano uno zero, quali proposizioni siano dimostrabili, sono problemi diversi, ma fondamentalmente analoghi.\\
La logica matematica prende le mosse da un problema spropositato: trasformare l'intera matematica in un gioco di scacchi, e stabilirne - decretarne - le regole. Non è per tracotanza che alcuni pensatori concepirono questo obiettivo, ma per necessità, o quasi per spavento.
Infatti, si può tollerare che la fondatezza  di metodi come la teoria degli insiemi di Cantor sia argomento - e nutrimento - di speculazione filosofica, sol finché le implicazioni di questa dottrina si limitano all'indagine di un buffo concetto di infinito.
Quando matematici di spicco riconoscono la rilevanza dei metodi insiemistici per la matematica nel suo complesso - Hilbert: ``Aus dem Paradies, das Cantor uns geschaffen\ldots'' - allora è necessario che si giunga ad un consenso sulla correttezza di questi metodi.\\
Di fronte alla controversia, la matematica reagisce nel solo modo che conosce: occorre capire cosa sia precisamente una dimostrazione, studiare le proprietà di questi oggetti, e dimostrare che non è possibile dimostrare una contraddizione, e, forse, avendo fortuna, 
dimostrare addirittura che una proposizione può essere dimostrata precisamente quando questo è impossibile per la sua negazione. Ecco, sintetizzato in maniera un po' puerile - ma avete la mia parola che non so fare meglio - il \vocab{programma di Hilbert}. Chiaramente, mancano mille
dettagli - il più importante: quali metodi sono consentiti nell'esecuzione del piano? Qual è la \vocab{metateoria} su cui si deve basare la dimostrazione della solidità delle fondamenta di ogni altra dimostrazione? Possiamo concederci di accantonare questa domanda.
Se il programma di Hilbert si potesse portare a termine in una metateoria - una qualunque, che non sia contraddittoria - questa costituirebbe un insieme di principi, e si lavorerò poi per scremarli. È chiaro, però, che non si deve barare - se si vuole dimostrare che un sistema assiomatico
$T$ è solido, non vale partire da una metateoria $MT$ che ha, fra i suoi assiomi, l'enunciato ``$T$ è solido''. Intuitivamente, perché l'intera operazione abbia un senso, è meglio che $MT$ sia - a prima vista, almeno - non meno affidabile di $T$ stessa.
Tecnicamente, il minimo che si possa pretendere è che $MT$ sia un sottoinsieme di $T$. In conclusione, il programma di Hilbert richiede, come minimo, di trovare un sistema formale abbastanza vasto da servire come ragionevole fondamento della matematica, e di identificare un segmento di questo sistema che 
sia intuitivamente valido, e capace di dimostrare la coerenza del sistema nel suo complesso.\\
Tutti sanno che il programma di Hilbert è deragliato a causa dei \vocab{teoremi di incompletezza di Gödel}, del 1931. È andata così: in pratica, un'operazione di hacking. Gödel ha considerato un arbitrario sistema assiomatico $T$,
sotto la condizione che abbia una presentazione effettiva e che sia capace di esprimere una modica quantità di aritmetica - ragionevolmente, qualunque teoremi si voglia prendere a fondamento della matematica deve avere queste caratteristiche. Questo sistema $T$, dimostra Gödel, è in grado di esprimere 
proposizioni a proposito di un calcolatore universale, e, in questo calcolatore universale, si può implementare un sistema formale qualunque, per esempio $T$ stesso. Pare, ora, di essere sulla giusta via per il programma di Hilbert: abbiamo $T$, e dentro $T$
c'è un pezzo di $T$ che è in grado di esprimere proposizioni a proposito delle dimostrazioni di $T$, questo pezzo vorrà essere $MT$.
Se si vuole seguire il programma di Hilbert, anzi, si deve passare di qua. Qui, però, cominciano i guai. Grazie a un trucco geniale, è possibile sfruttare questa situazione per costruire una proposizione aritmetica che non 
può né essere dimostrata né confutata in $T$: una \vocab{proposizione indecidibile}. Sfuma quindi la possibilità che $T$ permetta di dirimere ogni possibile questione matematica. Ma c'è di peggio: un'analisi accurata dell'argomento precedente rivela che la proposizione ``$T$ non è contraddittoria''
è indecidibile in $T$. A fortiori, quindi, nessuna metateoria che sia un sottoinsieme di $T$ - neppure, appunto, $T$ stessa - può dimostrare la non contraddittorietà di $T$.
Per l'arbitrarietà di $T$, il programma di Hilbert è rovinato.\\
Cosa abbiamo imparato da questo disastro?