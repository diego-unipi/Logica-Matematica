\section{Eliminazione dei quantificatori}

\begin{definition}[Formule equivalenti per una teoria]
    Siano $\varphi$ e $\psi$ $L$-formule e $T$ una $L$-teoria. Diciamo che $\varphi$ è \vocab{equivalenti} per $T$,
    denotato con $T \models \varphi \,\textcolor{red}{\leftrightarrow}\, \psi$ se $T \models \textcolor{red}{(} \varphi \rightarrow \psi \textcolor{red}{)} \land \textcolor{red}{(} \psi \rightarrow \varphi \textcolor{red}{)}$,
    ossia $T,\varphi \models \psi$ e $T,\psi \models \varphi$.
\end{definition}

\begin{definition}
    [Eliminazione dei quantificatori]
    La $L$-teoria $T$ ha l'\vocab{eliminazione dei quantificatori} se ogni $L$-formula è equivalente, per $T$,
    ad una formula senza quantificatori.
\end{definition}

Se una teoria coerente $T$ ha l'eliminazione dei quantificatori, e, per ogni formula chiusa \textcolor{red}{e senza quantificatori} $\varphi$, 
vale $T \models \varphi$ o $T \models \textcolor{red}{\neg} \,\varphi$, allora $T$ è completa.

Dimostreremo la seguente proposizione.

\begin{proposition}
    La teoria degli ordini lineari densi senza estremi $T_{oldse}$ ha l'eliminazione dei quantificatori.
\end{proposition}

\begin{corollary}
    $T_{oldse}$ è completa.
\end{corollary}

Vediamo la dimostrazione del corollario.

\begin{proof}
    $T_{oldse}$ è coerente perché $\QQ \models T_{oldse}$. Inoltre, nel linguaggio di $T_{oldse}$,
    non ci sono simboli di costante, quindi le uniche formule chiuse senza quantificatori sono 
    combinazioni booleane di $\textcolor{red}{\top}$ e $\textcolor{red}{\bot}$. È immediato che queste formule
    hanno un valore di verità definito.
\end{proof}

Altre teorie con l'eliminazione dei quantificatori sono, per esempio, $\Th(\CC,\textcolor{red}{0},\textcolor{red}{1},+,\cdot)$ e $\Th(\RR,\textcolor{red}{0},\textcolor{red}{1},+,\cdot,<)$:
rispettivamente, la teoria dei campi algebricamente chiusi e la teoria dei campi reali chiusi.

\begin{remark}
    Se $\varphi_1$ è una sottoformula di $\psi_1$, e rimpiazziamo una occorrenza di $\varphi_1$ in $\psi_1$ con una $\varphi_2$, che soddisfa $T \models \varphi_1 \leftrightarrow \varphi_2$,
    allora la formula $\psi_2$ ottenuta da questa sostituzione soddisfa $T \models \psi_1 \leftrightarrow \psi_2$.
\end{remark}

\begin{proof}
    [Dell'osservazione]
    Induzione strutturale.
\end{proof}

Introduciamo due nozioni di logica proposizionali che sono elementari ma spesso utili.

\begin{definition}
    [CNF e DNF]
    Una formula senza quantificatori $\varphi$ è in \vocab{forma normale congiuntiva (CNF)} se:
    \[ \varphi = \textcolor{red}{(}\, \alpha_{11} \, \textcolor{red}{\lor} \, \alpha_{12} \, \textcolor{red}{\lor} \, \dots \, \textcolor{red}{\lor} \, \alpha_{1n_1} \, \textcolor{red}{)} \, \textcolor{red}{\land} \, \dots \, \textcolor{red}{\land} \, \textcolor{red}{(}\, \alpha_{m1} \, \textcolor{red}{\lor} \, \alpha_{m2} \, \textcolor{red}{\lor} \, \dots \, \textcolor{red}{\lor} \, \alpha_{mn_m} \, \textcolor{red}{)}
    \]
    ovvero $\varphi$ una congiunzione di disgiunzioni di formule $\alpha_{ij}$ che possono essere formule atomiche o negazioni di formule atomiche. Simmetricamente,
    $\varphi$ è in \vocab{forma normale disgiuntiva (DNF)} se:
    \[ \varphi = \textcolor{red}{(}\, \alpha_{11} \, \textcolor{red}{\land} \, \alpha_{12} \, \textcolor{red}{\land} \, \dots \, \textcolor{red}{\land} \, \alpha_{1n_1} \, \textcolor{red}{)} \, \textcolor{red}{\lor} \, \dots \, \textcolor{red}{\lor} \, \textcolor{red}{(}\, \alpha_{m1} \, \textcolor{red}{\land} \, \alpha_{m2} \, \textcolor{red}{\land} \, \dots \, \textcolor{red}{\land} \, \alpha_{mn_m} \, \textcolor{red}{)}
    \]
    ovvero $\varphi$ una disgiunzione di congiunzioni di formule $\alpha_{ij}$ che possono essere formule atomiche o negazioni di formule atomiche.
\end{definition}

\begin{lemma}
    Data una formula $\varphi$ senza quantificatori, esistono $\psi_1$ in CNF e $\psi_2$ in DNF equivalenti a $\varphi$ per la teoria vuota (e quindi per ogni teoria).
\end{lemma}

\begin{proof}
    [Esercizio]
\end{proof}

\begin{proof}
    [$T_{oldse}$ ha l'eliminazione dei quantificatori]
    Per induzione strutturale, e sfruttando il fatto che $\models \textcolor{red}{(\forall\,x_k}\, \varphi\textcolor{red}{)} \to \textcolor{red}{(\neg\exists\,x_k\,\neg}\,\varphi\textcolor{red}{)}$,
    vediamo che ci bastano, data una formula $\varphi$ senza quantificatori, trovare $\psi$ senza quantificatori tale che $T_{oldse} \models \textcolor{red}{(\exists\,x_k}\, \varphi\textcolor{red}{)} \,\textcolor{red}{\leftrightarrow}\, \psi$.
\end{proof}

\begin{exercise}
 Da mettere.
\end{exercise}

\subsection{Ultraprodotti}
Introduciamo una prima tecnica per costruire modelli. Usando gli ultraprodotti dimostreremo il seguente.

\begin{theorem}
    [Compattezza - versione semantica]
    Data una $L$-teoria $T$ e una $L$-formula $\varphi$, se $T \models \varphi$ allora esiste un sottoinsieme finito $T' \subseteq T$,
    tale che $T' \models \varphi$.
\end{theorem}

Ovvero se $\varphi$ è conseguenza logica di un insieme di premesse $T$, allora basta, in realtà, una quantità finita di queste premesse per implicare $\varphi$.
Poco da stupirsi se si pensa alla conseguenza logica come dimostrabilità: infatti una dimostrazione \textcolor{purple}{- non abbiamo ancora formalizzato questo concetto -} è 
un argomento di lunghezza finita, non ha quindi spazio per riferirsi a più di una quantità finita di premesse. D'altro canto, sia l'enunciato sia la dimostrazione che vedremo sono puramente semantici:
la nozione di conseguenza logica che stiamo considerando è \textcolor{purple}{verità in tutte le strutture che soddisfano le premesse}.

Per mostrare un'applicazione squisitamente matematica degli ultraprodotti, dimostreremo altresì il seguente.

\begin{theorem}
    [Ax-Grothendieck]
    Sia $f : \CC^n \to \CC^n$ una funzione polinomiale iniettiva, allora $f$ è surgettiva.
\end{theorem}

Bando alle ciance.

\begin{definition}
    [Filtri ed ultrafiltri]
    Sia $I$ un insieme fissato, un \vocab{filtro su $\ps(I)$} è un sottoinsieme $F \subseteq \ps(I)$ tale che:
    \begin{enumerate}[(i)]
        \item $\emptyset \notin F$ e $I \in F$;
        \item $A \in F \land A \subseteq B \implies B \in F$;
        \item se $A,B \in F$ allora $A \cap B \in F$.
    \end{enumerate}
    Un filtro $U$ su $\ps(I)$ è un \vocab{ultrafiltro} se:
    \[ A \in U \lor I \setminus A \in U
    \]
\end{definition}

\begin{example}
    Fissato $I \ne \emptyset$ e $x \in I$, l'insieme:
    \[ U_x = \{ A \subseteq I \mid x \in A \}
    \]
    è un ultrafiltro, detto \vocab{ultrafiltro principale generato da x}. Nonostante il nome altisonante, questi ultrafiltri qui servono a poco.
    Noi abbiamo bisogno di ultrafiltri non principali.
\end{example}

\begin{example}
    Sia $I$ un insieme infinito. L'insieme:
    \[ F = \{ A \subseteq I : |I \setminus A| < \aleph_0 \}
    \]
    dei sottoinsiemi \vocab{cofiniti} di $I$ è un filtro su $\ps(I)$.
\end{example}

\begin{proposition}[Esistenza degli ultrafiltri]
    Se $F$ è un filtro su $\ps(I)$ allora esiste un ultrafiltro $U$ su $\ps(I)$ tale che $F \subseteq U$.
\end{proposition}

\begin{proof}
    Zorn.
\end{proof}

\begin{example}
    Sia $I$ un insieme infinito, allora esiste un ultrafiltro \textcolor{purple}{non principale} su $\ps(I)$. Basta infatti considerare 
    un ultrafiltro $U$ che estende il filtro dei cofiniti. Se $U$ fosse principale, allora esisterebbe un $x \in I$ avremmo $\{x\} \in U$,
    ma anche $I \setminus \{x\} \in U$ perché cofinito, da cui $\emptyset = \{x\} \cap (I \setminus \{x\}) \in U \; \lightning$.
\end{example}

\begin{definition}
    [Ultraprodotto]
    Fissiamo un linguaggio $L = (R,F)$ e una famiglia di $L$-strutture $\{M_i\}_{i \in I}$. Sia $U$ un ultrafiltro su $I$ e definiamo l'\vocab{ultraprodotto} $\prod_{i \in I} \faktor{M_i}{U}$, la $L$-struttura 
    che ha come dominio $\prod_{i \in I}D_i$ modulo la relazione di equivalenza:
    \[ \forall a,b \in \prod_{i \in I}D_i \quad a \sim b \iff \{ i \in I \mid a_i = b_i \} \in U
    \]
    L'interpretazione di $r \in R$ in questo dominio è:
    \[ \textcolor{purple}{([a_1],\ldots,[a_k]) \in \faktor{r}{U}} \overset{\text{def}}{\iff} \{ i \in I \mid (a_{1i},\ldots,a_{ki}) \in r^{M_i} \} \in U
    \]
    L'interpretazione di $f \in F$ è:
    \[ \textcolor{purple}{\faktor{f}{U}([a_1],\ldots,[a_k])} \overset{\text{def}}{=} [\{f_{M_i}(a_{1i},\ldots,a_{ki})\}_{i \in I}]
    \]
\end{definition}

\begin{remark}
    Occorre verificare che la definizione è ben posta, ossia che se $a_1 \sim_U a_1',\ldots a_k \sim_U a_k'$ allora:
    \[ ([a_1],\ldots,[a_k]) \in \faktor{r}{U} \iff ([a_1'],\ldots,[a_k']) \in \faktor{r}{U}
    \]
    ovvero:
    \[ \faktor{f}{U}([a_1],\ldots,[a_k]) \sim_U \faktor{f}{U}([a_1'],\ldots,[a_k'])
    \]
\end{remark}
